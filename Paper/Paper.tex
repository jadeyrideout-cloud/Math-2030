\documentclass{article}
\usepackage[left=1in,right=1in]{geometry}
\usepackage{amsmath}
\usepackage{graphicx}
\usepackage[colorlinks=true, allcolors=blue]{hyperref}
\usepackage{cite}
\usepackage{ragged2e}
\usepackage[title]{appendix}

\title{Your Paper}
\author{You}

\begin{document}
\maketitle

\begin{abstract}
Your abstract.
\end{abstract}

\section{Introduction and Questions}
\subsection{Definition of Pieces}
The standard European originates from Italy, and was originally made from ivory and sometimes ebony \cite{Rules}. Modern dominoes are usually made of porcelain, as are the set used in our research. European dominoes are rectangular, with dimensions of \(1\times2\). Each end of the domino holds a numerical value on each end in the number of pips, divided by a centre line as seen in the figure below.
\begin{figure}[h!]
    \centering
    \includegraphics[width=0.1\linewidth]{photos/Double 6.png}
    \caption{A double six domino.}
    \label{domino}
\end{figure}
Dominoes with two of the same number are called doubles, and are of utmost importance to our research. European dominoes come in sets of 'double 6', 'double 9' and 'double 12', the number indicating the highest value piece in the set, such as the double six seen in figure \ref{domino}. The number of pieces in a set is given by:
\begin{equation}
    \frac{N(N+1)}{2}+N
    \label{Eqn2}
\end{equation}
Where \(N\) is the double number. This formula is derived from \(\binom{N+1}{2}\), as there is one piece for every combination of two numbers between zero and the double number. An additional \(N\) is added to equation \ref{Eqn2} to account for the double pieces.

\subsection{Current Literature}
Current literature on European dominoes is rather scattered, being in many different fields. The bulk of studies use the form factor of a domino for other purposes, such as an educational tool. Assigning facts, questions, for simple problems to each end of a piece has been found to be a useful learning aid for younger students \cite{education}. Mathematical literature usually takes one of two forms. The most common is to define a game with similar rules but different pieces \cite{squares}. As seen below, many use squares with multicoloured edges for ease of tiling the plane. %figure
These dominoes are most often used in discussion of tiling or machine learning \cite{turing}. The other common form of mathematical research is the strategy of one specific regional variant of the game \cite{regional}. Most of these games use European dominoes, though the regionally of the rule set limit both reach and application. Our research aims to open an alley for a more standardised discussion on both of these topics.

\subsection{Research Questions}
To begin to fill this large gap in literature, we must first define the game of dominoes. The most popular form is Block dominoes, as outlined in section 2.1. This game is not of mathematical interest outside of probability, as all dominoes are placed in straight lines (see figure \ref{Block}). To broaden the area of study into fields such as tiling, our research questions arise.
\begin{enumerate}
    \centering
    \item What rule changes must be made to the game of Block dominoes to make it of mathematical interest?
    \item How and when do these rule changes begin to impact the playability of the game? 
\end{enumerate}
The first of these questions is fairly self explanatory, as rules can be added and changed to create different patterns. Question two on the other hand is qualitative in nature, and we believe to be just as important. In our eyes, further research into a traditional game that is not enjoyable to participate in is a waste of both time and energy, and thus we will attempt to find a rule set that strikes a balance between these two factors.

\section{methods}
Our main method to investigate question one was through real gameplay trials with three variations on the standard rules of dominoes. Though before modifying the rules of block dominoes, we first must define them in their basic form. %check book

\subsection{Rules of Block Dominoes}
There are many variations of games played with European dominoes, though the most common is Block Dominoes (hereby known as Block). This is the game used by international tournaments, and originates from late \(17^{th}\) century Britain \cite{Rules}. The following description is based off that given by Hoyle \cite{Rules}. The game is typically played with four players, each being dealt seven pieces. Play begins with whichever player was the double six, placing it in the centre of the play field. Play continues counter clockwise, each player placing a piece of a matching number of pips next to another, as seen in figure \ref{Block}
\begin{figure}[h!]
    \centering
    \includegraphics[width=0.5\linewidth]{photos/Block final.png}
    \caption{The end state of a game of Block}
    \label{Block}
\end{figure}

Dominoes must be placed in straight lines as seen above, with doubles being placed vertically. If at any point a player can not make a move, they must skip their turn. Play continues until one player has emptied their hand of dominoes, who is considered the victor. Further rules on scoring do exist, though are omitted for our purposes. \\\\A slight modification was made to these rules for ease of data collection, as we were unable to arrange tests with a group of four. For less than four players five pieces are dealt to each, with the leftover pieces kept as a 'draw pile.' When a player has no valid moves, they draw a domino from this pile. They may immediately play the drawn piece if an opening is available, otherwise their turn is skipped.

\subsection{Data collection}
Collection of data was conducted in two sessions, with a group of three and two respectively. Many rounds of each rule set were played, taking photographs of the final board state at the end of each round. These were then analysed by hand to create the tables shown in A. The data recorded includes the winner of each round, total dominoes placed, number of intersections of three and four dominoes, number of doubles placed, and the number of 'turns'. These categories are mostly self explanatory, though for our purposes 'turns' has a special definition. A 'turn' is any placement of a domino that deviates from the rules of Block, as defined in each rule set.

\subsection{Complexity}
To compare the complexity of each rule set, we defined a complexity value per domino for a round, \(C\), as the following:
\begin{equation}\label{Eqn1}
    C=\frac{10TI}{BN}
\end{equation}
Where \(T\) is the number of turns, \(I\) is the total number of three and four way intersections, \(B\) is the number of doubles placed, and \(N\) is the total pieces placed. As \(T\) and \(I\) raise complexity, they are directly proportional to \(C\). \(C\propto\frac{1}{B}\) as the number of doubles will always artificially increase the complexity of any possible rule set, and as play starts with the double six, \(B\neq0\). The factor of ten was added to the equation for ease of comparing values. As both \(T\) and \(I\) will equal zero for a round of Block, it's complexity will always be zero. This is our baseline for comparing complexities. 

\subsection{Rule Set I}
This first variation is the most similar to Block, adding one additional move. A piece can now be played on the end of a double, as shown in the figure below. 
\begin{figure}[!h]
    \centering
    \includegraphics[width=0.5\linewidth]{photos/A8(1) final.png}
    \caption{The end state of a game of rule set I}
    \label{gameA}
\end{figure}

From there, play continues in a straight line, though additional branches can be made off of further doubles.

\subsection{Rule Set II}
Rule set two all the rules of set I, adding the ability to play pieces at a ninety degree angle from the previous as seen in figure \ref{gameB}
\begin{figure}[h!]
    \centering
    \includegraphics[width=0.5\linewidth]{photos/B7 final.png}
    \caption{The end state of a game of rule set II}
    \label{gameB}
\end{figure}

Due to this, two new illegal moves must be defined. The first is the 'parallel placement' of pieces shown below, as to make this rule set less convoluted. The second is that the minimum space between to dominoes to not be touching is \(\frac{1}{2}w\). As the dominoes are not confined to a perfect grid, some small amount of shifting can occur at each intersection. This can compound, resulting in two dominoes that should be touching not physically doing so. Due to the way doubles shift alignment of the next piece by \(\frac{1}{2}w\), the minimum space between non adjacent dominoes will be this offset.

\subsection{Rule Set III}
Set III inherits all the rules of set II, adding an extra rule for intersections. Any intersection can now be treated as a double, placing a domino of equal value on an open side of the intersection as in figure \ref{gameC}
\begin{figure}[h!]
    \centering
    \includegraphics[width=0.5\linewidth]{photos/C5 final.png}
    \caption{The end state of a game of rule set III}
    \label{gameC}
\end{figure}

The only new illegal move is to place a double at one of these intersections.

\section{Results and Analysis}
Our qualitative results will be broken up by rule set, as each had unique findings. All raw data collected can be found in appendix \ref{A}, with graphs of select data found throughout this section.

\subsection{Rule Set I}
Our first rule set was found to be the most strategic. This is due to the limited amount of opportunities to place dominoes on any player's turn, raising the probability of a player having to skip said turn. After a few rounds had been played participants also started keeping track of each other's hands, known as 'counting cards', and playing to force other players to skip a turn. The data from several select rounds can be seen in the graph below:
\begin{figure}[h!]
    \centering
    \includegraphics[width=0.7\linewidth]{graphs/Rules I chart.png}
    \caption{Select data from rule set I}
    \label{graphA}
\end{figure}
Most metrics for rule set I stayed consistent between rounds, having the middle round length of the three.

\subsection{Rule Set II}
Rule set II was found by participants to be the most fun of the three. The ability to place pieces at ninety degree angles made for more interesting patterns as seen in figure \ref{gameB}. The method of blocking moves by counting cards became more involved, as moves could be physically blocked such as the double zero in the aforementioned figure. As expected the number of turns increased for this rule set, as seen below:
\begin{figure}[h!]
    \centering
    \includegraphics[width=0.7\linewidth]{graphs/Rules II chart.png}
    \caption{Select data from rule set II}
    \label{graphB}
\end{figure}
Round length for this rule set is more random than the first, with the number of turns increasing dramatically due to the possibility of one at every intersection.

\subsection{Rule Set III}

\subsection{Complexity}
Average and bounds of \(C\) values are given in the following chart:
\begin{figure}[h!]
    \centering
    \includegraphics[width=0.7\linewidth]{graphs/C chart.png}
    \caption{A bar chart of \(C\) values for each rule set.}
    \label{graphcom}
\end{figure}
As seen, average \(C\) values increase with each rule set. This increase appears to be linear, though no concrete conclusion can be drawn from such a limited data set. The maximum bound of \(C\) increase drastically between each rule set as is expected with the raising rule set, though the minimum bounds show an interesting finding. The lowest bound for both rule set I and II is zero, meaning at least one round was identical in play to a round of Block. This is due to the limited number of move options in any given game state in these two rule sets, giving a random chance of no player being able to branch out from the form of Block. This chance is greatly lowered in rule set III, as seen by the increased minimum bound.

\section{Conclusion}

\section*{Acknowledgements}
We would like to acknowledge several people if not for whom this research would not have been possible. Our biggest thank you goes to Kaitlin Healey, along with Tiffany Anne and Nicole Regier of the MUN Folklore Society for participation in data collection. The domino set used for data collection was a childhood set graciously donated by Lisa Adey-Rideout, which we appreciate greatly. A thank you also goes to Logan Sacrey, for being our first introduction to the wonders of \LaTeX.

\section*{Appendix}
\appendix
\section{Data}\label{A}
In all of the following, a line break separates session one and two.
\subsection{Rule Set I}
\begin{figure}[h!]
    \centering
    \includegraphics[width=0.5\linewidth]{tables/Rules I data.png}
    \label{A1}
\end{figure}

\subsection{Rule Set II}
\begin{figure}[h!]
    \centering
    \includegraphics[width=0.5\linewidth]{tables/Rules II data.png}
    \label{A2}
\end{figure}

\subsection{Rule Set III}
\begin{figure}[h!]
    \centering
    \includegraphics[width=0.5\linewidth]{tables/Rules III data.png}
    \label{A3}
\end{figure}

\subsection{C Values}
\begin{figure}[h!]
    \centering
    \includegraphics[width=0.5\linewidth]{tables/C data.png}
    \label{A4}
\end{figure}

\bibliographystyle{acm}
\bibliography{sources}

\end{document}
