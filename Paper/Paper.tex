\documentclass{article}
\usepackage[left=1in,right=1in]{geometry}
\usepackage{amsmath}
\usepackage{graphicx}
\usepackage[colorlinks=true, allcolors=blue]{hyperref}
\usepackage{cite}
\usepackage{ragged2e}

\title{Your Paper}
\author{You}

\begin{document}
\maketitle

\begin{abstract}
Your abstract.
\end{abstract}

\section{Introduction and Questions}
\section{methods}
Our main method to investigate question one was through real gameplay trials. Though before modifying the rules of block dominoes, we first must define them in their basic form. %check book
\subsection{Rule Set A}
This first variation is similar
\subsection{Rule Set B}
\subsection{Rule Set C}

\section{Results and Analysis}
\section{Conclusion}

\section*{Acknowledgements}
We would like to acknowledge several people if not for whom this research would not have been possible. Our biggest thank you goes to K. Healey, along with T. Anne and N. Regier of the MUN Folklore Society for participation in data collection. The domino set used for data collection was a childhood set graciously donated by L. Adey-Rideout, which we appreciate greatly. A thank you also goes to L. Sacrey, for both lending AV equipment, and being our first introduction to the wonders of \LaTeX.

\bibliographystyle{science}
\bibliography{sample}
%https://ia801405.us.archive.org/8/items/B-001-002-771/B-001-002-771.pdf (page 202 block rules)

\end{document}
