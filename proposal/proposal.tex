\documentclass{article}
\usepackage[top=0.5in,left=1in,right=1in]{geometry}
\usepackage{amsmath}
\usepackage{graphicx}
\usepackage[colorlinks=true, allcolors=blue]{hyperref}
\usepackage{ragged2e}

\title{Project Proposal}
\author{Jay Adey-Rideout}

\begin{document}
\maketitle

\section{Overview}
Our research will be based on the game of block dominoes, to be played with European style dominoes. Block dominoes is the game used in tournaments internationally, and is not very mathematically interesting. Dominoes can only be placed in straight lines, with no opportunity for branching out. We plan to analyse three different modifications to the rules of block dominoes to determine the answers to out two research questions. We hope our findings will be useful in opening a new area of research in both game theory and tiling, as literature on European dominoes is scarce.

\section{Research Questions}
Our two main research questions are as follows:
\begin{enumerate}
    \item How many simple rule changes must be made to block dominoes to make the game of mathematical interest?
    \item At what point do these rule changes effect the playability of the game?
\end{enumerate}
The first of these questions will be solved via the analysis as outlined in section 3, though the second is qualitative in nature. We believe question 2 is as important as the first, especially in research beyond this proposal, as research into a game that is not enjoyable would be a waste of time and resources.

\section{Methods}
The groundwork for our method is to define the rules of block dominoes, then expand them to make three different rulesets. Data will be collected as described in section 4, and will be compared by a round's \(C\) value. This value is a measure of complexity per domino, given by $$C=\frac{I_{tot}-D}{N}$$ With \(I_{tot}\) being the total number of three and four way intersections, D being the number of double dominoes, and N the total number of dominoes. Some qualitative analysis will also be performed to answer question 2.

\section{Data Collection}
Data collection Will be done through two sessions with small groups. Games of each rule set will be played, with images taken of the end state and data tracked in a spreadsheet. Data tracked includes the winner of a game, number of tiles placed, and numbers of different types of intersections. Analysis will be done after all data is collected, with some graphs produced. Qualitative remarks on each rule set will be gathered from all participants to help define research question 2


\bibliographystyle{science}
\bibliography{sample}

\end{document}
